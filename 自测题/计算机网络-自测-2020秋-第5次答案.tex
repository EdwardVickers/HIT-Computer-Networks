%% 基础配置
\documentclass[11pt]{article}
\usepackage[margin=1in]{geometry}
\usepackage[UTF8]{ctex}
\usepackage{amsmath}
\usepackage{listings}

\title{自主评测5}
\date{}
\author{}

%% 正文开始
\begin{document}
\maketitle

\begin{enumerate}
    \renewcommand{\labelenumii}{(\arabic{enumii})}
    %% 注意offset以8字节为单位 4000B的分组长度包含了IP头的20字节
    \item 3 \quad \\ 123456 \quad 0 \quad 1 \quad 1500 \quad 0 \\
                     123456 \quad 0 \quad 1 \quad 1500 \quad 185 \\
                     123456 \quad 0 \quad 0 \quad 1040 \quad 370
    \item
    \begin{enumerate}
        \item 202.118.1.0 \quad 255.255.255.192 \quad 62 \quad 202.118.1.1 \quad 202.118.1.62 \\
              202.118.1.64 \quad 255.255.255.192 \quad 62 \quad 202.118.1.65 \quad 202.118.1.126
        \item 子网1分配给局域网1,子网2分配给局域网2。
        \begin{table}[htbp]
            \centering
            \begin{tabular}{|c|c|c|c|} 
            \hline
            目的网络IP地址 & 子网掩码 & 下一跳IP地址 & 接口 \\ 
            \hline
            202.118.1.0 & 255.255.255.192 & - & E1 \\ 
            \hline
            202.118.1.64 & 255.255.255.192 & - & E2 \\ 
            \hline
            202.118.3.2 & 255.255.255.255 & 202.118.2.2 & L0 \\ 
            \hline
            0.0.0.0 & 0.0.0.0 & 202.118.2.2 & L0 \\
            \hline
            \end{tabular}
            \label{table-1}
            \caption{R1的路由表}
        \end{table}
        %% 注意路由聚合
        \item R2的路由表中到局域网1和局域网2的路由:
        \begin{table}[htbp]
            \centering
            \begin{tabular}{|c|c|c|c|} 
            \hline
            目的网络IP地址 & 子网掩码 & 下一跳IP地址 & 接口 \\ 
            \hline
            202.118.1.0 & 255.255.255.128 & 202.118.2.1 & L0 \\ 
            \hline
            \end{tabular}
            \label{table-2}
            \caption{R2的路由表中到局域网1、2的表项}
        \end{table}
    \end{enumerate}
    \item 255.255.255.224 \quad 30
\end{enumerate}

\end{document}
