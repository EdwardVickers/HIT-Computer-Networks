%% 基础配置
\documentclass[11pt]{article}
\usepackage[margin=1in]{geometry}
\usepackage[UTF8]{ctex}
\usepackage{amsmath}
\usepackage{listings}

\title{自主评测6}
\date{}
\author{}

%% 正文开始
\begin{document}
\maketitle

\begin{enumerate}
    \renewcommand{\labelenumii}{(\arabic{enumii})}
    \item
    \begin{enumerate}
        \item 在AS1中,子网153.14.5.0/25和子网153.14.5.128/25可以聚合为子网153.14.5.0/24;在AS2中,子网194.17.20.0/25和子网194.17.21.0/24可以聚合为子网194.17.20.0/23,但缺少194.17.20.128/25;子网194.17.20.128/25单独连接到R2的接口E0。 \\ 因此得到以下的路由表:
        \begin{table}[htbp]
            \centering
            \begin{tabular}{|c|c|c|} 
                \hline
                目的网络 & 下一跳 & 接口 \\ 
                \hline
                153.14.5.0/24 & 153.14.3.2 & S0 \\ 
                \hline
                194.17.20.0/23 & 194.17.24.2 & S1 \\ 
                \hline
                194.17.20.128/25 & - & E0 \\ 
                \hline
            \end{tabular}
            \label{table-1}
            \caption{R2的路由表}
        \end{table}
        \item 该IP分组的目的IP地址194.17.20.200与路由表中194.17.20.0/23和194.17.20.128/25两个路由表项均匹配,根据最长匹配原则,R2将通过E0接口转发该IP分组。
        \item R1与R2之间利用BGP4(或BGP)交换路由信息;BGP4的报文被封装到TCP协议分组进行传输。
    \end{enumerate}

    \item 根据距离矢量路由算法,收敛状态下各路由器的距离矢量为:
    \begin{table}[htbp]
        \centering
        \begin{tabular}{|c|c|c|c|} 
            \hline
            目的网络 & R1 & R2 & R3 \\ 
            \hline
            192.168.1.0/24 & 1 & 2 & 3 \\ 
            \hline
            192.168.2.0/23 & 3 & 2 & 1 \\ 
            \hline
        \end{tabular}
        \label{table-2}
    \end{table}

    当R3检测到子网192.168.2.0/23不可到达后,各路由器的距离矢量为:
    \begin{table}[htbp]
        \centering
        \begin{tabular}{|c|c|c|c|} 
            \hline
            目的网络 & R1 & R2 & R3 \\ 
            \hline
            192.168.1.0/24 & 1 & 2 & 3 \\ 
            \hline
            192.168.2.0/23 & 3 & 2 & 3 \\ 
            \hline
        \end{tabular}
        \label{table-3}
    \end{table}

    \newpage

    交换一次距离矢量后,各路由器的距离矢量为:
    \begin{table}[htbp]
        \centering
        \begin{tabular}{|c|c|c|c|} 
            \hline
            目的网络 & R1 & R2 & R3 \\ 
            \hline
            192.168.1.0/24 & 1 & 2 & 3 \\ 
            \hline
            192.168.2.0/23 & 3 & 4 & 3 \\ 
            \hline
        \end{tabular}
        \label{table-4}
    \end{table}

    第二次交换距离矢量后,各路由器的距离矢量为:
    \begin{table}[htbp]
        \centering
        \begin{tabular}{|c|c|c|c|} 
            \hline
            目的网络 & R1 & R2 & R3 \\ 
            \hline
            192.168.1.0/24 & 1 & 2 & 3 \\ 
            \hline
            192.168.2.0/23 & 5 & 4 & 5 \\ 
            \hline
        \end{tabular}
        \label{table-5}
    \end{table}
    
    R1所维护的距离矢量包括自身的距离矢量以及邻居R2最新交换过来的距离矢量。
\end{enumerate}

\end{document}
